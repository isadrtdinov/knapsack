\documentclass{article}
\usepackage[utf8]{inputenc}
\usepackage{indentfirst}
\usepackage{amsmath}
\usepackage{textcomp}
\usepackage{bm}
\usepackage{amsfonts}

\title{0-1 Knapsack solution}
\author{Sadrtdinov Ildus}
\date{March 2020}

\setlength{\parindent}{2em}

\begin{document}

\maketitle

\section{Introduction}

This paper describes an approximate solution for 0-1 Knapsack problem written in Rust. This solution consists of two parts, \textbf{Ibarra-Kim} algorithm, which is memory inefficient, and \textbf{Heuristic} algorithm, which is long in time. The combination of these two algorithms provides a powerful solution, which deals successfully with many cases and even strikes an optimal dynamic solution for some tests. The solution simply chooses the best answer from two mentioned algorithms.

\vspace{0.3cm}

This notation will be used further:

\begin{itemize}
    \item $W$ - the capacity of the knapsack
    \item $n$ - number of items
    \item $\{(c_i, w_i)\}_{i=1}^n$ - items to full the knapsack with costs $c_i$ and weights $w_i$
    \item $C_{greedy}$ - greedy solution cost
\end{itemize}

\section{Ibarra-Kim algorithm}

This algorithm outputs a $\frac{1}{\varepsilon}$-approximate solution for $0 < \varepsilon < \frac{1}{2}$. The approach of this method is to rescale item costs to range $\left[0, \frac{1}{\varepsilon^2}\right]$ with a simple formula:

$$
c_i^{*} = \frac{c_i}{2\varepsilon^2 C_{greedy}}
$$

\noindent
and to launch dynamic programming (DP) for rescaled costs. This results in \textbf{time complexity} of $\mathcal{O}\left(\frac{n}{\varepsilon^2}\right)$.

It turned out that DP with reduced memory doesn't work for costs (in the way that it works for weights). For this reason, DP maintains the full dynamic table, which results in $\mathcal{O}\left(\frac{n}{\varepsilon^2}\right)$ \textbf{memory cost}.

In final implemention $\varepsilon$ is set as follows:

$$
\varepsilon = 
\begin{cases}
    0.0195, & n \le 50000 \\
    0.04, & n > 50000
\end{cases}
$$

\section{Heuristic algorithm}

The heuristic algorithm is based on the greedy solution. The idea of the algorithm is rather simple: we definitely should take items with the greatest unit cost (the same as the greedy solution does) and we also should consider a list of candidates, which would fill the remaining space of the knapsack.

Denote item unit cost by $u_i = \frac{c_i}{w_i}$. Let $\sigma \in S_n$ be a permutation, which sorts items by unit cost, i.e. $u_{\sigma(1)} \ge u_{\sigma(2)} \ge \dots u_{\sigma(n)}$. We will take two parameters: $\alpha \in \mathbb{R}, 0 \le \alpha \le 1 $ and $m \in \mathbb{N}, m \le n $. The algorithm consists of several steps:

\begin{enumerate}
    \item Fill $\alpha W$ of knapsack's capacity with items with the greatest unit cost, i.e. find $j$ such that $\sum_{i = 1}^j w_{\sigma(j)} \ge \alpha W$. Items $\sigma(1), \dots, \sigma(j - 1)$ are taken to the knapsack.
    \item Let $W_{new} = W - \sum_{i = 1}^{j - 1} w_{\sigma(j)}$. Consider items $\sigma(j), \dots, ..., \sigma(j + m -1)$. Solve 0-1 knapsack problem for these items and capacity $W_{new}$.
\end{enumerate}{}

The last step is done by DP with reduced memory for weights.

\textbf{Time complexity}: $\mathcal{O}\left(m W_{new}\right) = \mathcal{O}\left((1 - \alpha) m W\right)$

\textbf{Memory cost}: $\mathcal{O}\left(W_{new}\right) = \mathcal{O}\left((1 - \alpha)W\right)$

\vspace{0.3cm}

Values for parameters used in the final implementation: $\alpha = 0.5, m = 1200$.

\section{Results}

The solution was tested in Yandex Contest.

\textbf{Time limit} - 10s, \textbf{memory limit} - 512 Mb.

\vspace{0.5cm}

\hspace*{-1.5cm}
\centering
\begin{tabular}{|c|c|c|c|c|c|c|c|}
    \hline
    \textnumero & $\bm n$ & $\bm W$ & \textbf{Greedy score} & \textbf{DP score} & \textbf{Solution score} & \textbf{Time} & \textbf{Memory} \\
    \hline
    1 & 50000 & 1892492 & 29433040 & 29448192 & 29448082 & 7.858s & 30.50Mb \\
    2 & 100000 & 577535 & 13472213 & 13475909 & 13475909 & 2.414s & 11.76Mb \\
    3 & 50000 & 1468889 & 11170413 & 11179617 & 11179616 & 6.079s & 24.04Mb \\
    4 & 100000 & 543896 & 4876185 & 4877302 & 4877300 & 2.286s & 10.55Mb \\
    5 & 50000 & 1237528 & 799761 & 1117491 & 1117344 & 3.258s & 466.03Mb \\
    6 & 100000 & 326301 & 199953 & 296293 & 296279 & 1.215s & 207.24Mb \\
    7 & 50000 & 2243398 & 33791873 & 33792261 & 33792261 & 9.166s & 31.16Mb \\
    8 & 100000 & 701824 & 12881119 & 12883188 & 12883188 & 2.9s & 10.88Mb \\
    9 & 50000 & 477920 & 528973 & 538688 & 538688 & 2.919s & 508.70Mb \\
    10 & 200000 & 198956 & 173206 & 214214 & 214213 & 1.673s & 492.90Mb \\
    \hline
\end{tabular}

\end{document}
